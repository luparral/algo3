\documentclass[10pt, a4paper,english,spanish]{article}
%\documentclass[10pt,a4paper]{article}
\usepackage[utf8]{inputenc} % para poder usar tildes en archivos UTF-8
\usepackage[spanish]{babel} % para que comandos como \today den el resultado en castellano
\usepackage[conEntregas]{caratula}
\usepackage{fullpage} %small margins
\usepackage[parfill]{parskip} %genera saltos entre parrafos
\usepackage{color}
\definecolor{gray}{gray}{0.35}
\usepackage{listings}
\usepackage{enumitem}
\usepackage{amsmath} %big brackets
\usepackage[pdftex]{graphicx}
\lstset{
    numbers=left,
    breaklines=true,
    tabsize=2,
    basicstyle=\ttfamily\color{gray},
}
\setlength{\parindent}{8pt}
\usepackage{mathtools}
\usepackage[margin=50pt]{geometry}
\usepackage{amsfonts}
\usepackage{flafter}
\usepackage{multicol}
\begin{document}

\materia{Algoritmos y Estructuras de Datos III}
\submateria{Trabajo Práctico Nro. 3}
\titulo{Heurísticas y Metaheurísticas}
\fecha{\today}
\integrante{Pablo Gomez}{156/13}{mago-1986@hotmail.com}
\integrante{Lucia Parral}{162/13}{luciaparral@gmail.com}
\integrante{Emanuel Lamela}{21/13}{emanuel93\_13@hotmail.com}
\integrante{Petr Romachov}{412/13}{promachov@gmail.com}
\maketitle
\newpage

\tableofcontents

\newpage
% Comienzo Ejercicio 1
\section{Ejercicio 1}

\subsection{Introducción}
\noindent \underline{\textbf{Contexto}}

Estamos encarando una competencia, en la que dados distintos puentes colgantes y participantes, estos deben cruzar los puentes sin caerse, dado que bajo cada puente corre un río de lava hirviendo.
Estos puentes tienen una característica importante, y es que no todos sus tablones están en óptimas condiciones, estando algunos rotos, y haciendo que si alguien los pisa, caiga sin remedio hasta el río de lava que fluye debajo. Esta es la única forma de caer del puente. Afortunadamente, estos tablones rotos se encuentran marcados, de forma que quien este cruzando el puente, sepa si puede pisar o no un tablón.
Un requisito de la competencia es que los puentes solo pueden ser cruzados dando saltos, tanto para ingresar o salir del puente como para avanzar de tablón en tablón.
En cuanto a los participantes, debido a su estado físico, cada uno tiene una capacidad máxima de tablones que pueden cruzar de un solo salto.
Notar que si hay demasiados tablones rotos consecutivos, si un participante no tiene la capacidad suficiente de salto para cruzarlos a todos juntos, entonces ese participante no podrá cruzar el puente ya que indefectiblemente caerá si lo intenta.
El ganador de la competencia será quien haya cruzado todos los puentes, dando la menor cantidad de saltos.

\noindent \underline{\textbf{El problema a resolver}}

Debemos diseñar un algoritmo, que dado un participante, la capacidad máxima de tablones que puede cruzar de un solo salto (salto máximo), un puente, con un número n de tablones del mismo y un mapeo de los tablones rotos, indique si es posible cruzar el puente y de ser así, indique a qué tablones debería saltar para cruzarlo dando la menor cantidad de saltos posibles. El algoritmo deberá tener complejidad O(n), siendo n la cantidad total de tablones (rotos y no rotos) del puente colgante.
En el caso en el que el participante no pueda cruzar el puente, debe devolverse la palabra "no".
Así mismo, el programa debe soportar resolver muchas instancias de este tipo ingresadas.

\noindent \underline{\textbf{Ejemplos}}

\noindent Para los ejemplos denotaremos:

\textit{C}, a la cantidad de tablones que el participante en cuestion puede cruzar de un solo salto.\newline
\indent \textit{N}, a la cantidad de tablones (rotos y no rotos) del puente en cuestion, donde a cada tablon los enumeraremos entre 1 y N.

\begin{enumerate}[leftmargin=0.5cm]

\item C = 4, N = 8, donde los tablones rotos son: 1,3,6,8.

El algoritmo debería devolver: 3 4 7 9

\item C = 4, N = 17, donde los tablones rotos son el 2,3,4,5.

El algoritmo debería devolver: no, puesto que una vez en el tablón 1, el participante no puede pisar ningún tablón que su capacidad C le permite pisar.

\item  C = 5, N = 41, donde no hay tablones rotos.
                         
El algoritmo debería devolver: 9 5 10 15 29 25 39 35 40 42

\end{enumerate}
\newpage

\subsection{Desarrollo}

Como idea principal para la resolución del problema, partiremos de la siguiente propiedad del problema:
para cualquier tablón $\boldsymbol{n_0}$ en que se esté situado en un momento dado, hacer un salto de distancia \textbf{X}, es siempre mejor que un salto de distancia \textbf{J}, para todo \textbf{J\textless X}. Este razonamiento 'goloso' es válido ya que vale que luego de saltar hasta x, se puede saltar a todos los que se podría haber saltado desde j, y posiblemente, más aún.

Es decir, si notamos CS[i], donde 0\textless i$\leq$n, (con n = cantidad de tablones del puente), como la conveniencia de salto para un tablón i, vale que:

\textbf{(}$\boldsymbol\forall$ \textbf{j,x:nat, j\textless x) CS[j]} $\boldsymbol\leq$ \textbf{CS[x]}

Por lo dicho anteriormente, concluimos (y en el siguiente inciso, demostraremos) que saltar a la posición más lejana posible dentro del rango de salto k (propio del jugador), nunca sera peor que hacerlo a una posición anterior a ésta y más aún, será la óptima decisión.

Con el siguiente pseudocódigo explicamos las bases principales de nuestro algoritmo:

\noindent Sea \textbf{S} el salto m\'aximo posible.\\
Sea \textbf{C} la cantidad de tablones del puente pasado como par\'ametro (tablones v\'alidos y no v\'alidos).\\
Sea \textbf{saltos} el vector en el que se almacenar\'an las posiciones a las que se salte. Cuando se declara a este vector, se le reserva un tama\~no \textbf{C}, ya que a lo sumo pueden realizarse \textbf{C} saltos (uno por tabl\'on).\\
Sea \textbf{actual} última posición válida a la que se saltó.\\
Sea \textbf{tablon\_en\_revision} el tabl\'on que estoy revisando en el ciclo.\\

\begin{lstlisting}
	Si (S > C) // Caso 1
		agrego C+1 a Saltos
		encontre solucion y salgo
	Si no // Caso 2
		si (S < 1)	
			no hay solucion y salgo
	Si no // Caso 3
		para (i entre 0 y C)
			actualizo el tablon_en_revision
			si (i es posicion valida)
				actualizo salto_mas_largo_posible
			si (tablon_en_revision - actual == S)
				actualizo actual
				agrego actual a saltos
			si (actual + S > C)
				agrego C+1 a saltos
				encontre solucion y salgo

\end{lstlisting}

\noindent En el condicional de la línea 1, se chequea si el rango de salto S pasado como parámetro supera a la cantidad de tablones, es decir, si con un solo salto se puede recorrer todo el puente. Si ese es el caso, se sale del algoritmo.\\
En el condicional de la línea 2, se chequea si no hay rango de salto S, en ese caso, el problema no tiene solución.\\
En el último condicional (línea 7), se entra a un ciclo en el que se van a recorrer del 0 al C (sin incluir) todos los tablones, llevando un registro del tablón que se esta revisando en la i-ésima iteración (lo llamamos tablon\_en\_revision), así como también, cuál es la última posición válida hacia la que salté (en el pseudocódigo la llamaremos actual). Si tablon\_en\_revision es válido, se lo guarda como potencial candidato a salto, dado que, hasta el momento, es el salto más largo posible. Luego, se chequea si, dada la posición actual y el tablon en revisión, ya chequee todos los tablones a los que potencialmente podía acceder desde actual (línea 12), sí es así, debo guardar como salto realizado al potencial candidato a salto (en el pseudocódigo lo llamamos salto\_mas\_largo\_posible) en el vector en el que almaceno los saltos y luego, actualizar como actual a esta posición, ya que se convertirá en la última posición válida hacia la que salté. Por último, verificamos si con el próximo salto desde la posición actual se llegará a cruzar el resto del puente, es decir, si al dar mi próximo salto encontré la salida (línea 15). Si ese es el caso, agrego la posición de salida (la C+1) y salgo del ciclo.\\
Finalmente, devolvemos el vector de saltos y su tamaño como solución del problema, en el formato pedido.\\
Vale la pena mencionar que repetiremos este algoritmo una vez por cada instancia pasada como entrada.\\



\subsection{Correctitud}

Sea \textbf{P} un puente pasado como par\'ametro y \textbf{C} la cantidad de tablones del mismo, describiremos \textbf{Z}, una soluci\'on del problema, como el arreglo que contiene un máximo C de elementos x donde $\forall$ x $\in$ Z, P[x] = 0, es decir, es una posici\'on v\'alida a la que el participante salt\'o (tambi\'en nos referiremos a estas posiciones como a los "saltos").
Demostraremos la siguiente propiedad, llam\'emosla M:\\

\textbf{M} = para cualquier \textbf{a} tabl\'on v\'alido de un puente P, con a $\in$ \{1\,...\,n\} y un salto m\'aximo k que un jugador puede realizar, efectuar el salto desde a hacia p = max\{a\,...\,a+k\} (siendo p una posici\'on v\'alida del puente) genera una menor o igual cantidad de saltos en Z que cualquier otra posici\'on distinta de p elegida.\\

Esto es equivalente a afirmar que cumpliendo la propiedad \textbf{M}, una solución es \'optima para el problema planteado, es decir, $\vert$M(Z)$\vert$ $\leq$ $\vert$Z$\vert$. Esto es: la longitud de Z  \underline{si para Z vale la propiedad M} \textbf{es menor o igual} a la longitud de Z en otras condiciones. (Aclaraci\'on: por longitud nos referimos a la cantidad de saltos realizados) \\

Supongamos una solución \textbf{S} óptima, donde hay algún \textbf{a} para el cual no vale \textbf{M}, es decir, se escoge una opción de salto \textbf{Q} $\neq$ max\{a\,...\,a+k\}, donde \textbf{Q} es una posición válida.\\
Consideremos entonces ahora a \textbf{S'}, una solución para la misma instancia que posee hasta la posición \textbf{a} los mismos saltos, pero que en \textbf{a}, en vez de realizar un salto a \textbf{Q} se realizó un salto a la posición válida \textbf{P} = max\{a\,...\,a+k\}.\\
Llamaremos $R_{p}$ al rango de posiciones válidas accesibles desde \textbf{P} $\in$ \{P\,...\,P+k\} y $R_{q}$ al \textbf{S} al rango de posiciones válidas accesibles desde \textbf{Q} $\in$ \{Q\,...\,Q+k\}.\\
Sabemos que $R_{p}$ $\cap$ $R_{q}$ $\neq$ $\emptyset$ ya que por lo menos exite \textbf{P} que es posición válida que es accesible desde $R_{q}$ y pertenece a $R_{p}$, ya que \textbf{Q} $<$ \textbf{P} $\leq$ \textbf{Q+k} $\leq$ \textbf{P+k}.

	\begin{figure}[h]
		\begin{center}
		   \includegraphics[scale=0.50]{esquema.png}
		\end{center}
	\end{figure}

Entonces pueden darse los siguientes casos:\\

(a) Existe al menos un \textbf{x} tal que \textbf{x} no pertenece a $R_{p}$ $\cap$ $R_{q}$ y x $\in$ $R_{p}$, es decir, existe una posición que es accesible desde \textbf{P} pero no desde \textbf{Q}. Acceder a esta posición solo puede provocar que se generen en total menos o igual cantidad de saltos que en \textbf{S}, ya que, en caso de ser necesario acceder a estas posiciones, desde \textbf{S} se debería haber realizado un salto extra, en cambio en \textbf{S'}, estas posiciones son accesibles en el primer salto \textbf{P}.
En el caso de que no sea necesario accederlas, se mantendría la misma cantidad de saltos que en \textbf{S}, por lo que \textbf{S'} sería igualmente óptima.\\

(b)En el caso de que no exista \textbf{x} tal que \textbf{x} no pertenece a $R_{p}$ $\cap$ $R_{q}$ y \textbf{x} $\in$ $R_{p}$, $R_{p}$ contiene las posiciones válidas entre \{P\,...\,Q+k\}, por lo que como teníamos \textbf{Q} $<$ \textbf{P} $\leq$ \textbf{Q+k}, se sigue que $R_{p}$ esta incluído en $R_{q}$, entonces obtendremos la misma cantidad de saltos en \textbf{S'} que en \textbf{S}, ya que en \textbf{S}, desde \textbf{P} el próximo salto válido será \textbf{Q+k}, que era el próximo salto válido desde \textbf{Q} en \textbf{S}. De esta forma, mantengo igualada la cantidad de saltos: de \textbf{a} a \textbf{Q} y de \textbf{Q} a \textbf{Q+k} en \textbf{S} y de \textbf{a} a \textbf{P} y de \textbf{P} a \textbf{Q+k} en \textbf{S'}.

Quedan así verificados todos los casos, demostrándose que una solución que cumple con la propiedad \textbf{M} es óptima.\newpage

%%%%%%%%%%%%%%%%%%%%%%%%%%


%%%%%%%%%%%%%%%%%%%%%%%%%%%%%%%%%%%%%%%%%%%%

\subsection{Complejidad}
\noindent Sea \textbf{S} el salto m\'aximo posible.\\
Sea \textbf{C} la cantidad de tablones del puente pasado como par\'ametro (tablones v\'alidos y no v\'alidos).\\
Sea \textbf{saltos} el vector en el que se almacenar\'an las posiciones a las que se salte. Cuando se declara a este vector, se le reserva un tama\~no \textbf{C}, ya que a lo sumo pueden realizarse \textbf{C} saltos (uno por tabl\'on).\\
Sea \textbf{actual} última posición válida a la que se saltó.\\
Sea \textbf{tablon\_en\_revision} el tabl\'on que estoy revisando en el ciclo.\\

\begin{lstlisting}
	Si (S > C) // Caso 1
		agrego C+1 a Saltos
		encontre solucion y salgo
	Si no // Caso 2
		si (S < 1)	
			no hay solucion y salgo
	Si no // Caso 3
		para (i entre 0 y C)
			actualizo el tablon_en_revision
			si (i es posicion valida)
				actualizo salto_mas_largo_posible
			si (tablon_en_revision - actual == S)
				actualizo actual
				agrego actual a saltos
			si (actual + S > C)
				agrego C+1 a saltos
				encontre solucion y salgo

\end{lstlisting}

\noindent Dependiendo de los par\'ametros de entrada S y C se entrar\'a por el Caso 1, Caso 2 o Caso 3 (notemos que entrar se puede entrar por un y solo un caso).\\
Tanto el Caso 1 como el 2 se basan en operaciones elementales que se ejecutan una cantidad constante de veces, por lo tanto, en ambos casos la complejidad es O(1).\\
En el Caso 3 tenemos como operaci\'on principal un ciclo for que se ejecuta a lo sumo C veces (puede llegar a ejecutarse menos de C veces si en alguna de las iteraciones, por el condicional de la l\'inea 15, se sale del ciclo).\\
Luego, tenemos todas operaciones elementales, sea cual sea el caso de que valga cualquier combinaci\'on posible de los condicionales de las l\'ineas 10, 12 o 15.\\
Finalmente, el Caso 3 tiene complejidad (en peor caso) de C operaciones elementales, es decir, C*O(1). Entonces, la complejidad temporal del Caso 3 es O(C).\\
Podemos ver entonces que la complejidad temporal del algoritmo termina siendo:\\
\begin{itemize}
\item[•]O(1) si se entra por el Caso 1 o 2.
\item[•]O(C) si se entra por el Caso 3.\\
\end{itemize}
Por lo anterior, en peor caso, la complejidad temporal del algoritmo propuesto es O(C).\newpage

%\thispagestyle{empty}
\subsection{Experimentación}
\noindent Para el proceso de experimentación del problema se plantearon distintos escenarios de test para corroborar que el algoritmo propuesto funcionara correctamente y que la cota de complejidad encontrada y justificada en la sección anterior, en la práctica, se cumpliera.\\

\noindent Llamamos escenario de test a un conjunto de pruebas que si bien son distintas, comparten alguna similitud.\\

\noindent Por ejemplo, un escenario es aquel en el cual un participante puede dar saltos de una distancia k constante, y las distintas pruebas del escenario variarían en el tamaño del puente y en el estado de sus tablones.\\

\noindent Dado que el CPU de la computadora utilizada para tomar los tiempos no está atendiendo únicamente a nuestro proceso, realizar una sola vez cada prueba podría darnos valores que no son cercanos a los reales. Para minimizar este margen de error, a cada prueba de cada escenario se la hizo ejecutar un total de 10.000 veces y se tomó el mejor valor. Notar que tomar el mejor valor no es una mala decisión, ya que mientras más chico sea el valor, más cerca estamos del valor real de tiempo que toma el algoritmo para una instancia dada.\\

\noindent En cada prueba se tomaron métricas para la posterior evaluación del algoritmo en la práctica. Notar que la medición no contempla tiempos de entrada/salida de datos, sino que contempla solamente el núcleo del algoritmo.\\

\noindent Para cada escenario testeado, se hicieron gráficos 2D que permitan ver de una manera más clara los resultados obtenidos en las pruebas del mismo. Estos fueron realizados con el software QitPlot que la cátedra proveyó.\\

\noindent En cuanto a qué casos testear, decidimos testear los casos “border” y casos aleatorios, mezclando en todos los casos distintas variaciones de saltos posibles, tamaño de los puentes y estado de los tablones de los mismos.\\

\noindent Los casos "border", son aquellos que están en los extremos de las capacidades del algoritmo, es decir, el mejor caso que el algoritmo puede resolver, y el peor. 
Para las instancias aleatorias, se diseñó un generador de estas, que dada una longitud, el estado de generaría los tablones se haría de forma aleatoria, es decir, dado un tablón t, en una prueba t puede ser válido y en otra no.  Este generador es capaz de generar múltiples instancias aleatorias.
Para todos los casos, se eligió una precisión de hasta 0,0001 ms (milisegundos). De ser menor, la notamos como 0.\\

\noindent En todos los casos se pudo comprobar que la práctica refleja lo expuesto en incisos anteriores.

\newpage \subsubsection{Escenario de mejor caso del algoritmo}

\noindent Para el algoritmo propuesto, el mejor caso de respuesta es el caso en el que la capacidad de salto del participante es superior a la longitud del puente, este caso tiene una complejidad de orden constante, es decir, el mejor caso es O(1).

\noindent Notar que para probar el mejor caso, la capacidad de salto no puede ser un parámetro variable, ya que es necesario que sea mayor que la cantidad de tablones.

\noindent Para evaluar el mejor caso, el generador de instancias, aplica como salto máximo, el tamaño del puente + 1. Por ende, para cada prueba realizada, si el puente contenía n tablones, el salto era de n+1.
Dicho esto, se generaron distintas instancias del algoritmo, con distintos tamaños, y con puentes de estado aleatorios, para verificar que el mejor caso no del estado de sus tablones, sino de que el salto sea mayor estricto que la cantidad de tablones.

\noindent Luego de realizar las pruebas pertinentes de este escenario, (y de realizar  10.000 veces cada una) como declaramos al inicio del insiso, mostramos los resultados obtenidos en el siguiente gráfico:

	\begin{figure}[h]
		\begin{center}
		   \includegraphics[scale=0.75]{casosDeTest/GRAFICOS/png/Ej1mejorCasoPNG.png}
		\end{center}
	\end{figure}

\indent En la figura superior se puede apreciar que el tiempo de resolución es constante a pesar de variar el tamaño de entrada y el estado de los tablones (generados aleatoriamente).\\

\subsubsection{Escenario de peor caso del algoritmo}

\noindent Para el algoritmo propuesto, el peor caso de respuesta, es decir, el caso que más tiempo demanda ejecutar,  es aquel en el que el participante puede cruzar todo el puente (por lo que el algoritmo debe continuar hasta el final del mismo) y dando saltos mínimos de avance, es decir, saltos de longitud 1.

\noindent Este caso, le toma al algoritmo recorrer todo el puente, por lo que, si n es la cantidad de tablones, el algoritmo es O(n).

\noindent Notar que en este caso tampoco tiene sentido variar la longitud del salto ya que si no, no estaríamos evaluando el peor caso.

\noindent A diferencia del mejor caso, aquí tampoco podemos variar el estado de los tablones, ya que estos deben ser todos válidos, para que, dando saltos de 1 de longitud, el participante pueda cruzar todo el puente, forzando al algoritmo a su peor caso.

\noindent Por ende, el único parámetro a variar para realizar pruebas, es la longitud del puente.\\

\noindent Luego de realizar las pruebas pertinentes de este escenario, (y de realizar  10.000 veces cada una) como declaramos al inicio del insiso, mostramos los resultados obtenidos en el siguiente gráfico:\\

	\begin{figure}[h]
		\begin{center}
		   \includegraphics[scale=0.75]{casosDeTest/GRAFICOS/png/Ej1peorCasoPNG.png}
		\end{center}
	\end{figure}

En la figura superior se puede apreciar que el tiempo de resolución de peor caso del algoritmo es lineal al tamaño del puente de entrada.

\subsubsection{Escenarios de casos promedio del algoritmo}

\noindent En estos escenarios, decidimos evaluar y tomar métricas para casos no tan “border”  como los de los escenarios anteriores. La idea es tomar muestras del algoritmo haciendo variar la longitud de salto del participante como el tamaño del puente, generando de manera aleatoria el estado de los tablones del mismo.

\noindent Para diferenciar bien los casos y poder analizar mejor, decidimos que cada escenario de las pruebas de caso promedio tenga una capacidad de salto constante. En cada uno de estos escenarios veremos cómo responde el algoritmo a medida que varía el tamaño de la entrada.

\noindent Notar que a diferencia  del mejor y del peor caso, aquí no está garantizado que el participante vaya a cruzar todo el puente. En los casos anteriores, cruzar el puente era requerido para poder contemplar el caso de análisis, tanto para el mejor como para el peor caso. Para poder ver en los gráficos, no solo el tiempo de respuesta del algoritmo en función de la entrada, sino también si el participante pudo o no cruzar el puente, se marcó de color verde a las instancias en las que sí pudo, y de color rojo a las que no.

\noindent Como era de esperar, como el estado de cada tablón es aleatorio, mientras más pequeño es el salto del participante, mayor es la probabilidad de que este no pueda cruzar el puente, ya que la probabilidad de que haya tantos tablones inválidos consecutivos como la capacidad de salto, aumenta.

\noindent Dicha probabilidad también aumenta cuando el puente crece, ya que hay un mayor espacio para que dicha secuencia de tablones inválidos aparezca.

\noindent Como fue mencionado anteriormente, para realizar estas pruebas, se diseñó un generador de instancias de caso promedio, que dado un tamaño de entrada y una capacidad de salto, generaba tantas instancias distintas como se quisiera, respetando el tamaño asignado. Eso nos permitió evaluar la respuesta del algoritmo para distintas instancias del mismo tamaño, y poder tomar un promedio.

\noindent Para estos casos, también cada prueba de cada escenario fue repetida un total de 10.000 veces para aminorizar el margen de error producido debido a que el CPU no está atendiendo únicamente nuestro proceso.

\noindent Como última observación a hacer, es interesante notar que para los casos en los que el participante no pueda cruzar todo el puente, los tiempos no siguen ningún patrón ni concordancia, ya que, si llamamos C a la capacidad de salto de un participante, el hecho de que haya C tablones rotos consecutivos es totalmente aleatorio ya que el estado de los tablones fue generado de manera aleatoria.

\noindent Sin más, presentamos los distintos escenarios.\\

\newpage \indent ESCENARIO DE SALTO = 2

	\begin{figure}[h]
		\begin{center}
		   \includegraphics[scale=0.75]{casosDeTest/GRAFICOS/png/randoms/ej1_random_salto2.png}
		\end{center}
	\end{figure}

Tal como lo expresado anteriormente, para saltos de baja longitud, la probabilidad de no poder cruzar el puente aumenta, por lo que los tiempos del algoritmo pueden variar mucho de instancia a instancia. \\ \\

\indent ESCENARIO DE SALTO = 4
	\begin{figure}[h]
		\begin{center}
		   \includegraphics[scale=0.75]{casosDeTest/GRAFICOS/png/randoms/ej1_random_salto4.png}
		\end{center}
	\end{figure}

Aquí se puede apreciar que una de las instancias generadas permitió al participante cruzar todo el puente, y que fue la instancia en la que más probabilidades tenía de hacerlo, es decir, la instancia con un puente de longitud = 10. Para el resto, continuaron apareciendo de manera aleatoria distintos baches en el puente que no permitieron al jugador terminar de cruzar el mismo. Notar que los baches se hicieron presentes al poco tiempo de recorrer el puente, tal como era esperado probabilísticamente.\\

\newpage \indent ESCENARIO DE SALTO = 8
	\begin{figure}[h]
		\begin{center}
		   \includegraphics[scale=0.75]{casosDeTest/GRAFICOS/png/randoms/ej1_random_salto8.png}
		\end{center}
	\end{figure}

En saltos de tamaño 8, se continuó observando que la probabilidad de poder cruzar el puente en altas instancias continúa siendo muy baja y el algoritmo termina rápidamente debido a eso.  Aunque como cada instancia no depende de las demás, no podemos acotar al tiempo del algoritmo, pero si podemos hacerlo, justificando que cierta cota es válida para el caso promedio, para saltos menores o iguales a 8.\\

\indent ESCENARIO DE SALTO = 16
	\begin{figure}[h]
		\begin{center}
		   \includegraphics[scale=0.75]{casosDeTest/GRAFICOS/png/randoms/ej1_random_salto16.png}
		\end{center}
	\end{figure}

A partir de aquí, notamos que con las mismas longitudes de prueba que en los escenarios anteriores, no se produjo una caída del puente, en ninguna de las distintas instancias que se probaron para cada uno de los tamaños (10, 100,…,800.000). Vale destacar, que con un salto de longitud = 16, se cae en el mejor caso, para la instancia de 10 tablones, ya que puede cruzarla toda de una, haciendo que la probabilidad de caer del puente pase a ser nula. Lo mismo ocurrirá con luego con saltos mayores, al superar a la instancia de 100 tablones.

Si bien en todas las pruebas de este escenario el participante pudo cruzar, eso no escapa de la probabilidad de que la generación aleatoria pueda dar 16 tablones rotos consecutivos. Por lo que no podemos dar una cota fija, pero podríamos estimar que en el caso promedio, el puente puede ser cruzado, lo que da una complejidad de O(n) para el caso promedio, con n igual a la cantidad de tablones.
Es O(n) ya que el algoritmo debe recorrer todo el puente para terminar.\\

\newpage \indent ESCENARIO DE SALTO = 32
	\begin{figure}[h]
		\begin{center}
		   \includegraphics[scale=0.75]{casosDeTest/GRAFICOS/png/randoms/ej1_random_salto32.png}
		\end{center}
	\end{figure}

Aquí y en adelante, la probabilidad de que haya una cantidad de tablones rotos consecutivos igual a la cantidad del salto, es realmente baja, más allá del tamaño de la entrada; tendría que ser una entrada exageradamente grande para que la probabilidad aumente apenas un poco. Por ende de aquí en adelante, afirmamos que el caso promedio del algoritmo es O(n), (recordando que n es la cantidad de tablones del puente en cuestión).

Por lo dicho recién, en los siguientes escenarios era esperado que la respuesta del algoritmo sea líneal al tamaño de la entrada.\\ \\

\indent ESCENARIO DE SALTO = 64
	\begin{figure}[h]
		\begin{center}
		   \includegraphics[scale=0.90]{casosDeTest/GRAFICOS/png/randoms/ej1_random_salto64.png}
		\end{center}
	\end{figure}

\newpage \indent ESCENARIO DE SALTO = 128
	\begin{figure}[h]
		\begin{center}
		   \includegraphics[scale=0.90]{casosDeTest/GRAFICOS/png/randoms/ej1_random_salto128.png}
		\end{center}
	\end{figure}

\indent ESCENARIO DE SALTO = 256
	\begin{figure}[h]
		\begin{center}
		   \includegraphics[scale=0.90]{casosDeTest/GRAFICOS/png/randoms/ej1_random_salto256.png}
		\end{center}
	\end{figure}


\subsubsection{Algunas conclusiones}
Haciendo un análisis probabilístico de nuestro algoritmo, por lo antes explicado podemos afirmar que en saltos grandes, el tiempo de computo de una instancia es lineal al tamaño del puente de la misma.
En cuanto a saltos pequeños no hay un patrón que se mantenga ya que el “hueco” sin tablones válidos por los cuales no se podría cruzar es totalmente al azar, pero probabilísticamente, si el salto es chico, dicho “hueco” se hará presente al poco tiempo de computo, por lo que el algoritmo responde eficientemente y podríamos acotar este tiempo para un caso promedio.
% Fin Ejercicio 1

\newpage
% Comienzo Ejercicio 1
\documentclass[10pt,a4paper]{article}
\usepackage[utf8]{inputenc} % para poder usar tildes en archivos UTF-8
\usepackage[spanish]{babel} % para que comandos como \today den el resultado en castellano
\usepackage{fullpage} %small margins
\usepackage[parfill]{parskip} %genera saltos entre parrafos
\usepackage{color}
\definecolor{gray}{gray}{0.35}
\usepackage{listings}
\usepackage{enumitem}
\usepackage{amsmath} %big brackets
\usepackage{mathtools}
\lstset{
    numbers=left,
    breaklines=true,
    tabsize=2,
    basicstyle=\ttfamily\color{gray},
}
\setlength{\parindent}{8pt}
\usepackage{mathtools}
\usepackage[margin=50pt]{geometry}
\usepackage{amsfonts}
\usepackage{flafter}
\usepackage{multicol}

\begin{document}

\section{Algoritmo exacto}
\subsection{Desarrollo}
El problema que se nos presenta es el de la k-PMP. El mismo trata sobre, dado un grafo G=(V,E) con pesos en las aristas, encontrar una particion de los nodos en k (o menos) conjuntos, tal que sea la partición de menor peso.
Lo que hace al peso de una partición, es la suma de los pesos de las aristas intrapartición (sin importar aquellas que no lo son). Una arista se dice \textit{intrapartición} cuando los 2 nodos sobre los que incide pertenecen al mismo conjunto de partición.\\

Dicho esto, para la realización del algoritmo exacto, optamos por revisar todas las particiones posibles de los n nodos, y quedarnos con la mejor (la de menor peso) de las que tengan k o menos conjuntos.
Se elaboraron podas y estrategias previas al analisis de todas las posibles particiones para mejorar los tiempos de ejecución, y mejorar la complejidad espacial.\\

\textbf{\underline{\textit{Nota:}} A lo largo de todo el desarrollo del algoritmo exacto, cuando se hace referencia a\\ \textbf{\textit{"particionesDeN"}} es una forma abreviada de hablar de las particiones del conjunto \{1,2,3,...,n\}} \\

El algoritmo \textit{\textbf{sin podas ni estrategias}} para generar las particiones de un conjunto de n nodos es el siguiente:

Si Miramos las particiones de un conjunto de 1, 2, y 3 elementos:

\begin{figure}[h]
	\begin{center}
	   \includegraphics[scale=0.7]{ParticionesNegro.png}
	\end{center}
\end{figure}
Y si notamos la siguiente similitud entre la transición de uno a otro (ejemplo, transición de particionesDe2 a \\particionesDe3):\\
\begin{figure}[h]
	\begin{center}
	   \includegraphics[scale=0.7]{ParticionesNegroRojo.png}
	\end{center}
\end{figure}\\
Veremos que las particionesDe3 son las particionesDe2 solo que agregando el 3  primero en un subconjunto aparte, y luego dentro de los subconjuntos de 2. Entonces:

\newpage
Para generar las particiones un conjunto de 3 elementos, a partir de las particiones de un conjunto de 2 elementos, hace 2 pasos:\\
\begin{enumerate}
\item Toma todas las de 2 elementos y le agrega un nuevo conjunto que solo contiene al 3 (en el gráfico, las 2 primeras particiones), ese nuevo grupo de particiones ya formará parte de las particiones finales de 3 elementos.\\
Esta cantidad de particiones añadidas cumple que \#particionesAñadidas = \#particionesTotalesDeN-1
\begin{figure}[h]
	\begin{center}
	   \includegraphics[scale=0.6]{explicacionPaso1.png}
	\end{center}
\end{figure}\\


\item Luego, vuelve a mirar las particiones de 1..2, y por cada una (llamemosle X) realizará lo siguiente:\\
\indent Añadirá una partición por cada subconjunto de X, y en cada una de ellas, agregará el valor de n (3 en este caso) a un subconjunto distinto para luego formar parte de las particiones finales buscadas.\\
\begin{figure}[h]
	\begin{center}
	   \includegraphics[scale=0.6]{explicacionPaso2.png}
	\end{center}
\end{figure}\\

\indent \textbf{\underline{Ej:}} Toma la partición \textbf{[1] [2]}:\\
\indent -Aplica el \textbf{3} al subconjunto \textbf{[1]}, quedando: \textbf{[1,3] [2]} que ya formará parte de las particiones finales.\\
\indent -Agrega a las particiones finales.\\
\indent -Revierte el 3 aplicado.\\
\indent -Luego aplica el \textbf{3} al subconjunto \textbf{[2]} de la misma partición, resultando en el conjunto: \textbf{[1] [2,3]}\\
\indent -Agrega a las particiones finales.

Y así se construyó la tercer y cuarta partición; repitiendo el procedimiento para la partición \textbf{[1,2]} se obtendrá la última partición de las 5.\\
\end{enumerate}
Esa es la \textbf{base} del algoritmo exacto utilizado para generar las particiones de un conjunto de números de tamaño n. Se deja guardada en el código la partición base, que es la única partición de un conjunto de 1 elemento, y para las demás se realiza el procedimiento mencionado n-1 veces, generando así las particiones de un conjunto de n elementos a partir de las de un conjunto de n-1.\\
Tiene una gran \textit{'pinta'} recursiva, pero decidimos hacerlo iterativo.\\

A continuación, el pseudocódigo del algoritmo (nuevamente, sin las podas):\\

\textbf{subConjunto} es un vector\textless Enteros\textgreater \\
\indent \textbf{Particion} es un vector\textless subConjunto\textgreater\\
\indent \textbf{PARTICIONES} es un vector\textless Particion\textgreater

\newpage
\underline{Recibe:}\\
-Un conjunto de particiones que esta vacío para que lo llene.\\
-El n
\begin{lstlisting}
getPartitionsOfN(n, PARTICIONES)

		PARTICIONES := {{1}}
	
		for i = 2 to n inclusive do
			lengthAnterior := PARTICIONES.length

			for j = 0 to lengthAnterior inclusive do
				jPartition := copyOf(PARTICIONES[j])

				PARTICIONES[j] U {i}			
					
				for p = 0 to jPartition.length hacer
					jPartition[p].push_back(i).
					PARTICIONES U {jPartition}.
					jPartition[p].pop_back();
				Fin Para
			
			end for
	
		end for

		devolver PARTICIONES;

end of getPartitionsOfN
\end{lstlisting}
---------------------------------------------------------------------------------------------------------------------------------------------------

\underline{\textbf{Las podas y estrategias:}}

\begin{enumerate}
\item \textbf{Poda de cantidad de conjuntos.}\\
Dado un n, la cantidad máxima de conjuntos que puede tener una particion de n es n. Pero las particiones de n tales que tienen mas de k conjuntos no nos interesan, ya que estamos en el problema de la k-PMP donde nos interesan las particiones que tengan a lo sumo k conjuntos.\\
Ejemplo utilizando las imágenes usadas para la explicación del algoritmo; en ese caso, n = 3. Si nuestro k (por ejemplo) es 2, ¿tendría sentido agregar la primer particion al conjunto de particiones final ([1][2][3])? No lo tendría puesto que por definición de una k-PMP no puede ser una solución del problema, y estaríamos generando particiones que luego analizariamos cuando ya sabemos de antemano que no son posibles.\\

\item \textbf{Poda de mejor peso encontrado hasta el momento.}\\
Esta poda se basa en que el peso de cualquier partición de n que tenga k o menos conjuntos es\textgreater = al peso de la solución óptima.
Dicho esto, si supieramos que alguna particion de n (y tamaño\textless = k) tiene peso X, entonces cuando estamos construyendo las particiones, si notamos que la particion que estamos construyendo ya sobrepasó ese límite (a pesar de que aún no tenga los n elementos), entonces no tiene sentido seguir construyendola ya que el peso intraparticion de los elementos ya aplicados lo superó y por lo tanto no es óptima. Si esto ocurre, se elimina esa partición de el conjunto de particiones credas hasta el momento.
Este valor X lo conseguimos mediante una \textit{cota inicial} que realizamos previo a la formación de las particiones de n, que será explicado en breve.\\

\item \textbf{Poda de actualización de mejor peso encontrado.}\\
En la poda recién explicada, se detallaba que dado un peso X calculado anteriormente, las particiones que se estén construyendo y alcanzaran dicho límite serían removidas. Pero ¿por qué no ir actualizando este límite si vamos encontrando particiones tales que ya tienen todos los nodos cargados y mejor peso que X?\\
Eso es lo que hace esta poda, si dada una partición U de n, tal que peso(U)\textless peso\_optimo\_parcial, entonces peso\_optimo\_parcial = peso(U). Esto permitirá que más particiones sean recortadas a futuro.\\
\textit{Aclaración:} Como se explicó, esta cota tiene utilidad recién cuando se encuentra una particion n. En nuestro algoritmo, no nos sirve mientras estamos construyendo las particiones de n-1, n-2,...,3,2 ya que necesitamos todos los nodos cargados para poder asegurar que encontramos un peso tal que peso\textless peso\_optimo\_parcial. Por ende somos conscientes que hasta la última iteración del for principal (pseudocódigo) no nos sirve, pero debido a que el peso de una partición actual en construcción ya lo calculamos para la poda anterior (poda número 2) realizar el checkeo y la asignación es O(1), así que no perdíamos nada por aplicarla.\\

\item \textbf{Estrategia previa para acotar el peso de la solución óptima.}\\
En esta estrategia, para obtener una cota superior de lo que puede llegar a pesar la solución óptima, decidimos ejecutar el algoritmo goloso antes del exacto (ejercicio 3 de este Trabajo Práctico) para que nos devuelva una configuración posible de los nodos (potencialmente óptima) y tomar su peso para usarlo como cota en la construcción de las particiones.\\
Si la cota devolviese la peor solución habría sido como si no hubiera existido, porque no hubiera acotado al problema.\\
\end{enumerate}

Dicho esto, a continuación el pseudocódigo \textit{con podas} de la parte del algoritmo que se encarga de la construcción de las particiones:\\

\begin{lstlisting}
getPartitionsOfN(pesos, n, k, PARTICIONES)

		PARTICIONES := {{1}}
	
		para i desde 2 hasta n inclusive hacer
			lengthAnterior := particiones.length

			para j desde 0 hasta lengthAnterior inclusive hacer		
				copy := copyOf(j-esima particion de PARTICIONES)

				Si la j-esima particion aun no tiene tamano igual a k
					agregar conjunto {i} a la j-esima particion			
				Sino
					Borrar la j-esima particion de PARTICIONES;
					j--;
					lengthAnterior--;
				Fin si
			
				Para p desde 0 hasta copy.length - 1 inclusive hacer
					copy[p].push_back(i);
					peso := peso(copy);
					
					Si peso < optimo_peso_parcial		
						Agregar copy a PARTICIONES
						Si i == n
							optimo_peso_parcial <- peso;					
						Fin Si
					Fin Si
					
					copy[p].pop_back();
					
				Fin Para
			
			Fin para
	
		Fin para

		return;

end of getPartitionsOfN
\end{lstlisting}
---------------------------------------------------------------------------------------------------------------------------------------------------
\newpage
\subsection{Análisis de la complejidad temporal}
Para la ejecución de la implementación del algoritmo (en C++), veremos el algoritmo en su peor caso, que es \textbf{sin ninguna cota aplicada}.\\
También, para los casos donde se realize el método push\_back() de la clase vector, si bien su costo es\\ \textbf{O(1) amortizado}, para simplificar el analisis de complejidad, se tomará como O(1).\\

Tenemos 3 ciclos a mirar.\\

\noindent \underline{El \textbf{primer ciclo for}}:\\
Itera un total de n-1 veces. Como es el for "padre", la complejidad temporal final será:

\textit{O(costo iteración 1) + O(costo iteración 2) + ... + O(costo iteración n-1)}. $\forall$\textit{n\textgreater =2}\\

\noindent Veamos el costo de cada iteración:

Se ejecuta solamente una asignación y \textbf{el segundo for}, por lo que la complejidad de cada iteración será igual a la complejidad de dicho segundo for en la misma iteración (la asignación es en O(1)). Veamoslá:\\

\noindent \underline{El \textbf{segundo ciclo for}}:\\
El for cicla tantas veces como la variable \textbf{lengthAnterior} indique.\\
Como dicha variable toma el tamaño de PARTICIONES, que al principio de cada iteración i del primer for contiene todas las particiones del conjunto \textit{\{1,2..i-1\}}, entonces la complejidad de dicho for será:

\textit{ O(\#particionesDeI-1 * costo de cada iteración del segundo for)}.

\noindent Veamos ahora el costo de cada iteración del segundo for.

\begin{enumerate}
\item Realizar el backup de una partición tiene costo \textit{O(n)} puesto que a lo sumo tiene a todos los nodos, es decir, n nodos.
\item Acceder a la j-esima posición de PARTICIONES y modificarlo agregandole un subConjunto tiene costo O(1).
\item Por último tenemos el costo del último for (tercer for), que itera los subConjuntos de la particion backupeada.
\end{enumerate}
Por lo que la complejidad temporal del segundo for en una iteración es:\\
\textit{O(n + costo del tercer for)}\\

\noindent \underline{El \textbf{tercer ciclo for}}:\\
Dada la j-esima partición backupeada en el segundo for, itera todos los subConjuntos. En el peor caso una partición puede tener n subConjuntos (1 nodo en cada uno), por lo que el costo es:\\
\textit{O(n * costo de cada iteración)}.

\noindent Donde el costo de cada iteración es:
\begin{enumerate}
\item Acceder a un conjunto y agregar un elemento: \textit{O(1)}.
\item Copiar la particion backupeada a PARTICIONES: \textit{O(n)}.
\item Acceder al último conjunto modificado y eliminar el elemento recién agregado: \textit{O(1)}.
\end{enumerate}
Costo final de cada iteración: \textit{O(n)}.\\

\noindent Por ende, el costo final \textbf{(de peor caso)} del tercer for es:

\textit{\textbf{O(n * n) = O($n^2$)}}\\ \\

Ahora, reemplazando la cota de complejidad conseguida del tercer for en el costo de \textbf{cada iteración del segundo for} nos queda:

\textbf{\textit{O(n + costo del tercer for) = O(n + $n^2$) = O($n^2$)}}

\noindent Volviendo a la complejidad del segundo for en una iteración i del primer for, teníamos:

\textit{O(\#particionesDeI-1 * costo de c/iteración del segundo for)}. = \textbf{\textit{O(\#particionesDeI-1 * $n^2$)}}

\newpage
Volviendo a la complejidad del primer for, teníamos que el costo total del algoritmo era:

\textit{O(costo iteración 1) + O(costo iteración 2) + ... + O(costo iteración n-1)}. $\forall$\textit{n\textgreater =2}\\
Donde \textit{costo iteración i} es el costo del segundo for en la iteración i del primer for.\\

Reemplazando lo conseguido del análisis del segundo for, el costo total del algoritmo es:

\textbf{\textit{O(\#particionesDe1 * $n^2$)) + O(\#particionesDe2 * $n^2$) + ... + O(\#particionesDeN-1 * $n^2$)}}.\\



\underline{Analizando la cantidad de particiones posibles distintas de un conjunto de N elementos.}\\
Los \textbf{\textit{Números de Bell}} (http://en.wikipedia.org/wiki/Bell\_number) son los números que indican la cantidad de particiones distintas posibles de un conjunto dado. Los mismos fueron presentados por \textit{Eric Temple Bell} y se definen bajo la siguiente fórmula \textbf{recursiva}:\\

INSERTE LATEX AQUI DE LA FORMULA DE BELL.\\

Se conocen distintas cotas superiores de estos números (http://en.wikipedia.org/wiki/Bell\_number\#Growth\_rate), y la que vamos a usar es la establecida por \textit{Berend. y Tassa. (Improved Bounds on Bell Numbers and on Moments of Sums of Random Variables)}, la cual acota superiormente al termino Bn de los números de Bell por la siguiente relacion:\\

INSERTE LATEX AQUI DE LA COTA.\\ \\


\noindent Luego, como la cantidad de particiones de un conjunto es estrictamente creciente en el tamaño del mismo, volviendo a donde habíamos quedado con respecto a la complejidad total del algoritmo:

\noindent \textit{O(\#particionesDe1 * $n^2$)) + O(\#particionesDe2 * $n^2$) + ... + O(\#particionesDeN-1 * $n^2$)} $\in$

\noindent \textit{O((n-1) * (\#particionesDeN-1 * $n^2$))} $\in$ 

\noindent \textit{O($n^3$ * \#particionesDeN-1)} $\in$ (Usando la cota mencionada)

\noindent \textit{O($n^3$ * $((n-1)/ln(n-1+1))^n$)}.\\

Finalmente, removiendo constantes:

\textbf{\textit{O($n^3$ * $(n/ln(n))^n$)}} que es una cota de la complejidad temporal final del algoritmo\\ \\ \\

\underline{\textbf{Análisis con las podas aplicadas}}

Las podas aplicadas al algoritmo en sí (dejando de lado la ejecución previa del algoritmo goloso) son las 3 primeras mencionadas anteriormente.

Para la \textbf{primer poda}, se agregó un if que en cada iteración del segundo for consulta por el tamaño actual de la partición para evitar revisar particiones de más de k conjuntos.\\
Dado que el método \textit{.size()} de vector es en \textit{O(1)}, la guarda del if se ejecuta en tiempo constante.\\
Si se ingresa por el if la ejecución continúa su flujo normal, de lo contrario, se hacen 2 operaciones básicas y un borrado, el cual en peor caso tiene costo \textit{O(n)}, por lo que no modifica la complejidad ya que previo a eso se había invertido O(n) en copiar la instancia, y porque el mayor costo lo tiene el tercer for con \textit{O($n^2$)}.\\
Eliminar dicha instancia evita que se siga ramificando apartir de ella un subárbol de desiciones que crece exponencialmente y que no tiene sentido revisar.\\
También, evitando generar esas particiones no solo ganamos temporalmente, sino también (en la práctica) en la espacialmente.

Para la \textbf{segunda poda}, se hace la medición del peso de la partición durante el tercer for antes de agregarla al conjunto de particiones final. Esta validación no se hace cuando se agrega un conjunto nuevo aparte (antes del tercer for) ya que agregar un conjunto con un solo elemento no modifica el peso de las aristas intrapartición.\\
Calcular el peso de la partición tiene costo \textit{O($n^2$)}, por lo cual, teóricamente estamos perdiendo complejidad, ya que cada iteración del tercer for ya no cuesta \textit{O(n)}, sino \textit{O($n^2$ + n) = O($n^2$)}, pero en la práctica favoreció de una manera muy positiva cuando trabaja en conjunción con la cota superior dada por el algoritmo goloso.

Finalmente para la \textbf{tercer poda}, se hace la actualización del peso\_optimo\_parcial. Esto se realiza mediante un if con una guarda que se ejecuta en \textit{O(1)}, al igual que el cuerpo del mismo.\\
Esto es así ya que el peso de la partición lo teníamos calculado y guardado cuando lo utilizamos en la segunda poda.\\

\noindent En conclusión, las podas no empeoraron la complejidad del algoritmo, a excepción de la poda numéro 2, que si bien, por el análisis teórico transforma \textit{O(n)} en \textit{O($n^2$)}, trabajando conjuntamente con el algoritmo goloso ejecutado previamente, es la poda que más eficiencia nos provió.\\
El caso en el que esta poda no tenga efecto positivo sobre el algoritmo y finalmente termine siendo una contra, sería el caso en el que el algoritmo goloso devuelva la peor partición posible de los nodos, lo cual tiene una ínfima posibilidad de ocurrir.\\

\noindent \textbf{\textit{Nota: }} En cuanto al algoritmo goloso no se hace un análisis del mismo puesto que eso ya fue hecho en el ejercicio 3 de este trabajo práctico.\\Además, este presenta una complejidad polinomial, la cual es despreciable en comparación con las escalas de complejidad que ronda el algoritmo exacto en sí.




  
\newpage
\subsection{Experimentación}
Para la experimentacion, se realizaron distintos escenarios de test. Cada escenario contiene valores que son producto de distintas tomas de datos del algoritmo en ejecución.

Como es sabido, la computadora en la que se realizaron las mediciones, no está atentiendo nuestro proceso únicamente. Es por eso, que realizar una única medición de cada instancia no nos asegura fidelidad.\\
\indent Para aminorizar esta falencia, se repitió cada instancia de cada caso de test de cada escenario, un total de 10 veces, y se tomó el mejor tiempo.\\
Notar que tomar el mejor tiempo (en lugar del promedio) no es una mala desición, ya que entre 2 mediciones de tiempos distintas, t1 y t2, si t1\textless t2, eso nos asegura que en t1 el procesador estuvo más focalizado en nuestro proceso, y por ende, es una medición más fiel del mismo.

Para realizar las mediciónes, se realizó un generador de instancias aleatorias, que dada una cantidad de nodos (n), una cantidad de aristas (m) y un entero (k), genera de forma aleatoria m aristas, tomando para cada una, 2 nodos de n, que no tengan ya una arista entre ellos.

También, para el escenario de un k dado, no se hicieron pruebas donde ocurra que k\textgreater =n, ya que la respuesta de estos casos es trivial, (que cada nodo podría ir en un conjunto distinto). Hacer las mediciones de un caso trivial no tiene mucho sentido, por lo que en todas las mediciones, siempre vale que k\textless n.

En todos los casos se midieron grafos de n nodos, que tengan una densidad de aristas del 25\%, 50\%, 75\% y 100\%.

Los valores de k elegidos fueron 3, 5, 8 y 13, que siguen la serie de \textit{Fibonacci} para una variación más uniforme.\\

\underline{\textbf{Los escenarios:}}
\textbf{\textit{ESCENARIO CON K = 3:}}
	\begin{figure}[h]
		\begin{center}
		   \includegraphics[scale=0.70]{graficos/k3.png}
		\end{center}
	\end{figure}\\
	\begin{figure}[h]
		\begin{center}
		   \includegraphics[scale=0.50]{graficos/tablak3.png}
		\end{center}
	\end{figure}\\

	
\newpage	
\textbf{\textit{ESCENARIO CON K = 5:}}
	\begin{figure}[h]
		\begin{center}
		   \includegraphics[scale=0.70]{graficos/k5.png}
		\end{center}
	\end{figure}\\
	\begin{figure}[h]
		\begin{center}
		   \includegraphics[scale=0.50]{graficos/tablak5.png}
		\end{center}
	\end{figure}\\
\newpage
\textbf{\textit{ESCENARIO CON K = 8:}}
	\begin{figure}[h]
		\begin{center}
		   \includegraphics[scale=0.70]{graficos/k8.png}
		\end{center}
	\end{figure}\\
	\begin{figure}[h]
		\begin{center}
		   \includegraphics[scale=0.50]{graficos/tablak8.png}
		\end{center}
	\end{figure}\\
	
\newpage
\textbf{\textit{ESCENARIO CON K = 13:}}
	\begin{figure}[h]
		\begin{center}
		   \includegraphics[scale=0.50]{graficos/k13ConZoom.png}
		\end{center}
	\end{figure}\\
\indent Si le alejamos un poco el zoom...
	\begin{figure}[h]
		\begin{center}
		   \includegraphics[scale=0.50]{graficos/k13SinZoom.png}
		\end{center}
	\end{figure}
	\begin{figure}[h]
		\begin{center}
		   \includegraphics[scale=0.50]{graficos/tablak13.png}
		\end{center}
	\end{figure}\\

\end{document}
% Fin Ejercicio 1

\newpage
% Comienzo Ejercicio 1
\section{Ejercicio 3}

\subsection{Introducción}

\noindent \underline{\textbf{Contexto}}

Una importante empresa de logística de sustancias debe llevar a cabo la tarea de transportar una cantidad determinada de químicos desde una fábrica hasta un depósito. Las sustancias a transportar tienen entre cada par de ellas, una propiedad llamada "peligrosidad".
Para realizar esta tarea, la empresa cuenta con una cantidad ilimitada de camiones con umbral determinado (e igual para todos los camiones) de peligrosidad, es decir, puede soportar hasta un cierta cantidad de sustancias en base a la peligrosidad que estas tienen entre sí.
La empresa quiere determinar cual es el mínimo número de camiones necesarios para transportar todos los productos sin que en ningún camión se supere el umbral de peligrosidad y además saber, en qué camion fue colocado cada producto.

\noindent \underline{\textbf{El problema a resolver}}

Dado n el número de productos a transportar, los coeficientes de peligrosidad entre cada par de productos i, j (con i $\neq$ j), $h_{ij}$ y M, la capacidad máxima de "peligrosidad" que un camión puede transportar, devolver una configuración que utilice la mínima de camiones necesarios para transportar todos los productos sin que en ningún camión la peligrosidad de los mismos exceda el umbral M y también devolver la cantidad de camiones que se utilizaron.

Un aspecto a tener en cuenta es que hay varias posibles configuraciones válidas posibles, que incluso requieran la misma cantidad mínima de camiones, siendo cualquiera de éstas una respuesta posible y correcta. Dado esto, se nos pide que utilicemos la técnica de \textit{Backtracking} inteligentemente para que sea lo más veloz posible.

\noindent \underline{\textbf{Ejemplos}}

\noindent Para los ejemplos denotaremos:

\textit{M}, al umbral de peligrosidad máxima que pueden soportar los camiones.\newline
\indent \textit{h(x,y) = h(y,x)}, a la función que define la peligrosidad entre 2 productos x e y.\newline
\indent \textit{h(C) = $\sum_{\substack{p_i,p_j \in C \\ i < j}} h_{i,j}$}, a la función que define la peigrosidad de un camión.

\begin{enumerate}[leftmargin=0.5cm]

\item Supongamos un M = 90 y que se nos da el siguiente conjunto de productos:

$\left\{ {p1, p2, p3, p4}\right\}$

\noindent Y sus peligrosidades:

h(p2,p1) = 10 \ \ h(p3,p2) = 10 \ \ h(p3,p1) = 10\\
h(p4,p3) = 30 \ \ h(p4,p2) = 20 \ \ h(p4,p1) = 10

\noindent La siguiente configuración es la que debería dar como salida el algoritmo:

Camión 1 = $\left\{ {p1, p2, p3, p4}\right\}$ $\implies$ \newline
\indent h(Camión 1) = h(p2,p1) + h(p3,p2) + h(p3,p1) + h(p4,p3) + h(p4,p2) + h(p4,p1) = 90

\noindent Notar que, al ser conjuntos y la peligrosidad no estar afectada por el orden, la siguiente también es correcta:

Camión 1' = $\left\{ {p4, p3, p2, p1}\right\}$ $\implies$ h(Camión 1) = h(Camión 1')
\bigskip
\item Supongamos un M = 50 y los mismo productos con sus respectivas peligrosidades del Ejemplo1. La configuración de 1 solo vehículo rompe las reglas de seguridad. En este caso, el algoritmo podría optar por organizar las sustancias de esta manera:

Camión 1 = $\left\{{p1,p2,p3} \right\}$ $\implies$ h(Camión 1) = h(p3,p2) + h(p3,p1) + h(p2,p1) = 30\\ 
\indent Camión 2 = $\left\{{p4} \right\}$ $\implies$ h(Camión 2) = 0

\noindent O bien así:

Camíon 1' = $\left\{{p1,p2} \right\}$ $\implies$ h(Camión 1') = h(p2,p1) = 10\\
\indent Camión 2' = $\left\{{p3,p4} \right\}$ $\implies$ h(Camión 2') = h(p4,p3) = 30

\noindent Cualquiera de estas 2 opciones que decida dar como resultado el algoritmo es considerada correcta, tanto como cualquier otra que utilice nada más 2 vehículos.

\end{enumerate}
\newpage
\subsection{Desarrollo}

Para resolver el problema planteado en el inciso 1, una posible opción era generar todas las posibles configuraciones de productos en camiones y luego ver a) si era una configuración válida, es decir, ningún camión superaba el umbral M y b) cuáles de las configuraciones válidas requerían usar la minima cantidad de productos y luego, elegir una de éstas.\newline
\indent Esta solución de fuerza bruta, donde se probaban todas las opciones posibles y luego se elegía la mejor fue descartada rapidamente, ya que la cantidad de configuraciones(válidas o no) equivale a la cantidad de particiones posibles de un conjunto de n productos, y ésta se representa con la formula recursiva de Bell donde:

$B_{n+1} = \sum\limits_{k=0}^n \binom {n} {k}B_k$

\noindent Este número de Bell crece exponencialmente, por lo que salta a la luz la necesidad de utilizar la técnica de backtracking para frenar la generación de soluciones inválidas y de soluciónes que, de antemano, sepamos que no van a ser óptimas.\\
\indent Backtracking es una técnica algorítmica que, recursivamente va construyendo candidatos a soluciones y abandona cada candidato parcial cuando determina que no va a poder completarse en una solución válida. En nuestro problema, al armar las distinas configuraciones, el algoritmo que presentamos irá generando recursivamente las particiones, pero siguiendo los lineamientos de backtracking, una vez que se detecta que la configuración ya no va a poder ser válida, esta configuración parcial es descartada, así como también, todas las configuraciones que iban a generarse a partir de ésta.

\bigskip
En este problema, lo que determina que una configuración que aún está incompleta sea considerada inválida es:
\begin{enumerate}

\item[I.] Algún camión superó por los productos contenidos en él el umbral de peligrosidad
\item[II.] Se ha superado la mínima cantidad de camiones necesarios para ser solución, es decir, previamente se había detectado que había otra configuración que requería una menor cantidad de camiones.

\end{enumerate}

De esta forma, vemos que usar la técnica de Backtracking genera la solución en un tiempo mucho mejor que al usar fuerza bruta, dado que no llega a generar una gran cantidad de candidatos a solución inválidos a priori, según los lineamientos I. y II., surgidos de la problemática presentada en el inciso 3.1). Vale recalcar que esa cualidad de tener 'candidatos a solución', en nuestro caso, subconjuntos de productos en camiones o 'particiones incompletas' junto con la posibilidad de testear si ésas pueden llegar a producir soluciones válidas, es lo que habilita el uso de esta técnica.

De todas formas, el algoritmo todavía dista de proveer una solución en tiempo polinomial, por lo que incluímos algunos intentos previos de acotar las posibles soluciones, de forma tal que el backtracking sea más efectivo. A estos fines, intentamos:

\begin{itemize}

\item Calcular previamente (y de una manera naive) cual es el número mínimo de camiones a partir del cuál es relevante buscar la solución. Caso contrario, el algoritmo de backtracking, más allá de las podas que realizara y que anteriormente describimos, debería considerar en un primer momento, que una solución con n camiones (con n cantidad de productos) es válida. Ejecutando este primer cálculo, acotamos la búsqueda antes de entrar a la función de backtracking, proporcionándole información extra, de forma tal que, en mejor caso (o caso promedio), descartará varias configuraciones parciales por ya poseer la información de que no son solución óptima.

\begin{lstlisting}[mathescape]
int dar_Cota_Inicial(Productos, Peligrosidades, Umbral M)
	res = 0, primeroACargar = 0, peligrosidadActual = 0
	
	Para todo i desde 0 hasta Productos.size
		Para todo j desde primeroACargar hasta i
			peligrosidadActual += Peligrosidad(Producto[i], Producto[j], Peligrosidades)
			Si peligrosidadActual >= M $\lor$ noHayMasProductos
				res++
				peligrosidadActual = 0
				primeroACargar = i
			Fin Si
		Fin Para
	Fin Para
	return res
Fin
\end{lstlisting}

Dado el pseudo código, uno puede observar lo 'naive' que se mencionaba anteriormente. Si este algoritmo \noindent preliminar al backtracking, que lejos está de chequear todas las posibilidades, devuelve una cantidad de camiones x, con 1$\leq$x\textless n, entonces el algoritmo de backtracking que hace un análisis exhaustivo de las posibilidades debe dar una solución s tal que 1$\leq$s\textless x.

\end{itemize}

Seguido de esto viene el paso de backtracking que efectivamente va a resolver el problema, pero con el agregado de la cota que nos provée el algoritmo recién mencionado. Este pseudo código ilustra la idea:

\begin{lstlisting}[mathescape]
void backtracking(Productos,Peligrosidades,M,CamionesActuales,OptimaCant,CamionesRes)
	//Nos fijamos si esta solucion parcial es valida y optima, luego si es una solucion final
	Si CamionesActuales.size $\leq$ OptimaCant $\land$ CheckUmbral(CamionesActuales,Peligrosidades,M)
		Si YaVimosTodosLosProductos
			OptimaCant = CamionesActuales.size
			CamionesRes = CamionesActuales
			return
		Fin Si
		
		//Faltan agregar Productos, pruebo el proximo producto en cada camion y en un camion nuevo.
		Producto x = Productos.ultimo
		Productos.SacarUltimo


		Para todo i desde 0 hasta CamionesActuales.size
			CamionesActuales[i].PonerAlFinal(x)
			backtracking(Productos,Peligrosidades,M,CamionesActuales,OptimaCant,CamionesRes)
			//Saco el producto para poner en el proximo camion
			CamionesActuales[i].SacarUltimo
		Fin Para
		
		//Lo pongo en un camion aparte
		Camion NuevoCamion
		Camion.PonerAlFinal(x)
		CamionesActuales.PonerAlFinal(NuevoCamion)
		backtracking(Productos,Peligrosidades,M,CamionesActuales,OptimaCant,CamionesRes)
	Si no //Estoy en un caso donde se supera el umbral o la cantidad de camiones supera la optima
		return
	Fin si
Fin
\end{lstlisting}

\newpage
\subsection{Complejidad}

Algunas aclaraciones previas al análisis de complejidad del ejercicio:

\begin{enumerate}

\item Producto es int, Productos es Vector(int), Camion es Vector(int), Camiones es Vector(Camion), Peligrosidades es Vector(Vector(int))
\item Asumo que las operaciones de vector PonerAlFinal(push back), sacarUltimo (pop back), Constructor por defecto (sin indicador de tamaño), el operator[], size y la asignación (por referencia) tienen complejidad $O(1)$. Cabe aclarar que esto no es siempre cierto para push back, pero para simplificar el argumento voy a tomarlo como tal.
\item La subrutina CheckUmbral(Camiones,Peligrosidades,M) comprueba que los camiones sean válidos, o sea que no excedan el umbral M, Camiones y Peligrosidades se pasan por referencia, al igual que el M. La función que caracteriza su complejidad pertenece a O($n^2$), donde $n$ son los productos dispersos por los camiones. La función responde a este pseudo codigo.
\bigskip
\begin{lstlisting}
bool CheckUmbral(Camiones,Peligrosidades,M)
	Para todo i desde 0 hasta Camiones.size
		int acum = 0
		Si Camiones[i].size > 1
			Para todo x desde 0 hasta Camiones[i].size - 1
				Para todo y desde x+1 hasta Camiones[i].size
					Producto px = Camiones[i][x]
					Producto py = Camiones[i][y]
					acum += Peligrosidad(px,py,Peligrosidades)
				Fin Para
				Si acum >= m
					return falso
				Fin Si
			Fin Para
		Fin Si
	return verdadero
Fin
\end{lstlisting}

\end{enumerate}

Dicho esto, el análisis de la complejidad del algoritmo de backtracking esta sujeta a esta función recursiva:

\[ T(i) = \left\{ \begin{array}{ll}
         (n-(i+1)+1)T(i-1) + O((n-i)^2) & \mbox{, si $0 < i < n$}\\
         O(n^2) & \mbox{, si $i = 0$}\end{array} \right. \]
Cancelando los índices:

\[ T(i) = \left\{ \begin{array}{ll}
         (n-i)T(i-1) + O((n-i)^2) & \mbox{, si $0 < i < n$}\\
         O(n^2) & \mbox{, si $i = 0$}\end{array} \right. \]       

Donde i es el número del próximo producto a ser puesto en los distintos camiones. Si el for principal del algoritmo esta sujeto a la cantidad de camiones, la pregunta es, ¿Por qué la función de recursión no lo está también? La respuesta yace en el análisis del peor caso. La máxima cantidad de llamados recursivos ocurre cuando cada camión tiene 1 solo producto, dado que se genera la mayor cantidad de camiones para dicho llamado recursivo. Si el próximo producto a ser analizado es el i, entonces los elementos i+1...n ya fueron colocados en distintos camiones, por ende la cantidad en peor situación es (n-(i+1)). Sumado a esto, ocurre un último llamado recursivo en el que se coloca el producto i en un nuevo camión, por ello el +1 multiplicando a T(i-1). El segundo término de T(i) describe la complejidad de hacer CheckUmbral en dicho paso. El caso base es hacer CheckUmbral con todos los químicos colocados.\newline
\indent Supongamos una cantidad n (con n $\geq$ 5 para el caso particular de esta demostración) de productos, esto implica que el último producto es $p_n-1$, o el producto (n-1), desarrollando la función desde (n-1) se obtiene lo siguiente:

$T(n-1) = (n-(n-1))T(n-2) + O((n-(n-1))^2) = 1T(n-1) + O(1) = 2T(n-3) + O(2^2) + O(1) = $

$2(3T(n-4) + O(3^2)) + O(2^2) + O(1) = 3!T(n-4) + O(2*3^2) + O(1) = $

$3!(4T(n-5) + O(4^2)) + O(2*3^2) + O(1) = 4!T(n-5) + O(3!*4^2) + O(2!*3^2) + O(0!*1^2) =$

$...(i = 0)... = (n-1)!T(0) + \sum\limits_{k=1}^{n-1} O((k-1)!*k^2) = (n-1)!T(0) + O(\sum\limits_{k=1}^{n-1} (k!*k))$

Por lo demostrado en el ejercicio 1.f de la Práctica 1, por propiedades de la función O y de la función factorial, nos queda que:

$(n-1)!T(0) + O(((n-1)+1)!) - 1 = (n-1)!O(n^2) + O(n!) - 1 = O((n-1)*n^2)) + O(n! - 1) =$

$O(n*n!) + O(n!)$

Además, como n $\in \aleph \implies ( n*n! \geq n! \iff n \geq 1)$ lo cual es cierto ya que n es un natural.
Por ende, $O(n*n!) + O(n!) = O(max\{n*n!,n!\}) = O(n*n!) = T(n-1)$ que es de donde arrancamos. Y $T(n-1)$ define la complejidad del algoritmo. Finalmente, el algoritmo tiene complejidad O($n*n!$). 

Esto no es todo ya que a esta complejidad hay que sumarle el costo temporal del algoritmo Dar\textunderscore Cota\textunderscore Inicial. El análisis de él es más simple. La peor circumstancia que puede ocurrir para Dar\textunderscore Cota\textunderscore Inicial es que todos los productos puedan concatenarse en un único camión, ya que para meter un producto i, tengo que obtener la peligrosidad (en $O(1)$) de i con respecto a todos los elementos desde 0...i-1 anteriores. Se harían $1 + 2 + 3 + ... + (n-1) + n = \frac{n * (n+1)}{2}$ (por Ejercicio 1.a, Práctica 1) pedidos de coeficientes de peligrosidad. El algoritmo es de orden temporal cuadrático.

En conclusión, si decimos que $f(n)$ es la función que caracteriza la complejidad del algoritmo que resuelve el problema, podemos decir que $f(n) = O(n*n!) + O(n^2)$. Sabemos, por el ejercicio 3.b de la Práctica 2, que dado r $\in \aleph$, $n^r \in O(n!) \implies n^r \in O(n*n!)$. Entonces, $f(n) = O(n*n!)$.

Si bien la cota de lo que podría tomar resolver el problema es holgada, recordemos que realmente se llega a ella en el caso en que ningún producto pueda viajar a la par de otro, ya que se deben descartar todas las particiones posibles antes de poder decir con seguridad que es la manera más eficiente de transportarlos.

\newpage
\subsection{Experimentación}
Para el proceso de experimentación del problema se plantearon distintas pruebas para corroborar que el algoritmo propuesto funcionara correctamente y que la cota de complejidad encontrada y justificada en la sección anterior, en la práctica, se cumpliera.

Al igual que en el ejercicio 1 Y 2, dado que el CPU de la computadora utilizada para tomar los tiempos no está atendiendo únicamente a nuestro proceso, realizar una sola vez cada prueba podría darnos valores que no son cercanos a los reales. Por lo que para minimizar este margen de error, a cada prueba se la hizo ejecutar un total de 750 veces (menos que las de los ejercicios 1 y 2 ya que la complejidad de este algoritmo no es polinomial) y se tomó el mejor valor. Notar que tomar el mejor valor no es una mala decisión, ya que mientras más chico sea el valor, más cerca estamos del valor real de tiempo que toma el algoritmo para una instancia dada.

En cada prueba se tomaron métricas para la posterior evaluación del algoritmo en la práctica. Notar que la medición no contempla tiempos de entrada/salida de datos, sino que contempla solamente el núcleo del algoritmo.

Se representó la información tomada mediante gráficos 2D que permitan ver de una manera más clara los resultados obtenidos en las pruebas. Estos fueron realizados con el software QitPlot que la cátedra proveyó.

Para el testeo, se diseñó un generador de instancias aleatorias, que dado un umbral y una cantidad de productos, genera aleatoriamente la peligrosidad entre los mismos. Dicho software es capaz de generar múltiples instancias que el algoritmo del ejercicio 3 resolvería todas juntas.

Con este software pudimos evaluar cuanto toma nuestro algoritmo para distintas instancias aleatorias del problema.

Para todos los casos, se eligió una precisión de hasta 0,0001 ms (milisegundos). De ser menor, la notamos como 0.

En todos los casos se mantuvo constante a M, y se probaron distintas cantidades de edificios, dejando de manera aleatoria la peligrosidad entre ellos, tal como fue explicado antes cuando se describió al generador de instancias del problema.
También aprovechamos los gráficos para realizar una comparación, entre lo que tarda el algoritmo normalmente, y lo que tarda cuando retiramos la "cota inicial" que pusimos en el desarrollo del programa para mejoras de performance.

A continuación presentamos los distintos gráficos en 2D que reflejan las pruebas realizadas. Para cada tamaño se realizaron pruebas con instancias distintas y las diferencias fueron muy sutiles (del orden de los microsegundos).\\

\indent ESCENARIO CON M = 2
	\begin{figure}[h]
		\begin{center}
		   \includegraphics[scale=0.50]{experimentos/random/graficos/2.png}
		\end{center}
	\end{figure}
	\begin{figure}[h]
		\begin{center}
		   \includegraphics[scale=0.50]{sincota/graficos/2.png}
		\end{center}
	\end{figure}



\newpage\indent ESCENARIO CON M = 16
	\begin{figure}[h]
		\begin{center}
		   \includegraphics[scale=0.50]{experimentos/random/graficos/16.png}
		\end{center}
	\end{figure}
	\begin{figure}[h]
		\begin{center}
		   \includegraphics[scale=0.50]{sincota/graficos/16.png}
		\end{center}
	\end{figure}

 
\newpage\indent ESCENARIO CON M = 64
	\begin{figure}[h]
		\begin{center}
		   \includegraphics[scale=0.50]{experimentos/random/graficos/64.png}
		\end{center}
	\end{figure}
	\begin{figure}[h]
		\begin{center}
		   \includegraphics[scale=0.50]{sincota/graficos/64.png}
		\end{center}
	\end{figure}


\indent ESCENARIO CON M = 256
	\begin{figure}[h]
		\begin{center}
		   \includegraphics[scale=0.50]{experimentos/random/graficos/256.png}
		\end{center}
	\end{figure}
	\begin{figure}[h]
		\begin{center}
		   \includegraphics[scale=0.50]{sincota/graficos/256.png}
		\end{center}
	\end{figure} \\


\newpage \subsubsection{Algunas conclusiones}
Para este ejercicio, nos sucedió que su complejidad dificultaba la experimentación, por lo que cada experimento demoró mucho tiempo, ya que el algoritmo es exponencial.\\
Hay dos cuestiones que, con el tiempo con el que dispusimos, intentamos analizar en las pruebas:
\begin{enumerate}
\item Diferencias entre el tamaño de la cantidad de umbrales: A pesar de que nuestra intuición nos llevó a pensar que las instancias de umbral muy pequeño iban a arrojar un peor resultado que las de umbral mayor, no pudimos comprobar ésto en las pruebas realizadas, ya que no llegamos a observar mayores diferencias en los resultados de las distintas instancias.\\
\item Diferencias de la performance del algoritmo al utilizar la función de dar cota Inicial: En este caso, nuevamente nuestra intuición nos llevó a pensar que dar la cota inicial podía llegar a reducir considerablemente el tiempo de performance de nuestro algoritmo, no pudimos comprobarlo empíricamente, ya que no pudimos comprobar que se mejoraran los tiempos en las instancias que testeamos.\\
\end{enumerate}
Sin embargo, una vez concluídos los experimentos que llegamos a realizar, y sin más tiempo de realizar otros posteriores, creemos que una buena idea hubiera sido realizar los gráficos utilizando una escala logarítmica, para poder apreciar mejor los cambios en la curva, ya que al ser exponencial y crecer tan rápidamente, estas variaciones resultaron impercibibles en los gráficos.
% Fin Ejercicio 1

\newpage
% Comienzo Ejercicio 1
\section{Heurística de búsqueda local}
\subsection{Desarrollo}
\noindent \textbf{\underline{Comentarios preliminares}}
\hfill \newline
\begin{itemize}
\item Si bien es cierto que para los conjuntos, como elementos matemáticos, no es coherente que exista una ''posición'' dentro de ellos, en otras palabras, una noción de orden de los elementos de los mismo, voy a asumir que hay una asignación implícita que está atada a la distribución de los nodos descripta por la solución. Es decir, un nodo tiene asignado un número $x$ en la solución $\iff$ el conjunto, dentro de la particion, al que pertenece se etiqueta como $x$.
\item El pseudo-código descripto más abajo es de bastante alto nivel para luego facilitar la lectura del código fuente.
\item El cálculo de los valores de ciertos índices (por ej: en qué conjunto estoy, en qué posición del vector estoy, entre otros) se realiza inmediatamente luego de entrar al while principal, en ambas vecindades. Se programó así para que haya menos código y sea más simple a la vista.
\item Solucion es un vector de ints, Pesos es una matriz de Peso, Peso es float, Particion es vector de vectores de nodos.
\end{itemize}
\hfill \newline
\textbf{\underline{Formalización de vecindades y pseudo-códigos}}
\hfill \newline

\textbf{Heurística de Búsqueda Local 1}

La vecindad, en este caso, va a estar definida por $N(S, numConjActual, posConjActual) = $ intercambiar, en la partición formada por $S$ que denotamos como $particionSolucion$, el nodo que corresponde al conjunto $numConjActual$ en la ''posición'' $posConjActual$ por algún otro nodo de conjuntos, de la misma partición, $numConjComparar$ tales que $numConjActual < numConjComparar$; \newline con $1 \leq numConjActual < numConjComparar \leq k$, $1 \leq posConjActual \leq \#(conjActual)$.

Para ilustrar parte de la idea:

\includegraphics[scale=0.75]{ejercicio4/Vecindad1.png}

Por ende, lo que debe hacer el algoritmo es, dada la ubicación actual $(numConjActual, posConjActual)$, debe buscar en los conjuntos, cuyas ''posiciones'' relativas a la particion son $\{(numConjActual+1)...k\}$, algún nodo que al intercambiarlo mejore el peso total de la partición. En otras palabras, un conjunto dentro de la partición tal que $conjActual'$ y $conjComparar'$, que son los conjunto resultado luego de intercambiar el nodo de $numConjActual$ en $posConjActual$ con algún nodo perteneciente a $conjComparar$, cumplan la siguiente desigualdad $peso(conjActual) + peso(conjComparar) > peso(conjActual') + peso(conjComparar')$.

Notar 2 detalles:

\begin{itemize}
\item No se busca intercambios en el mismo conjunto ya que el peso resultaría idéntico.
\item El último conjunto a explorar en busca de intercambios es el $k$, ya que no hay conjuntos ''delante'' del $k$.
\end{itemize}

\hfill \newpage
\textbf{Pseudo-código}

\begin{lstlisting}
busqueda_local_1(Solucion solucion, Int k, Pesos pesos)
	Particion particionSolucion(k)
	organizar los nodos en particionSolucion acorde a solucion

	si k > 1
		numConjActual = 0
		numConjComparar = 1
		posConjActual = 0
		posConjComparar = 0

		mientras numConjActual < k
			si (termine de recorrer el conjunto actual)
				buscar en (numConActual+1...k) el proximo conjunto no vacio y colocar su numero en numConjActual

				si numConjActual < k
					buscar en (numConjActual+1...k) el proximo conjunto no vacio y colocar su numero en numConjComparar

					si numConjComparar > k
						terminar ejecucion
					fin si

					posConjActual = 0
				si no
					terminar ejecucion
				fin si
			fin si

			si (termine de recorrer el conjunto comparar)
				buscar en (numConjComparar+1...k) el proximo conjunto no vacio y colocar su numero en numConjComparar

				si numConjComparar > k
					posConjActual++
					buscar en (numConjActual+1...k) el proximo conjunto no vacio y colocar su numero en numConjComparar

					si numConjComparar > k
						terminar ejecucion
					fin si
				fin si
			fin si

			posConjComparar = 0

			mientras ((no termine de recorrer conjActual) && (no termine de recorrer conjComparar))
				pesoConjActual = peso(particionSolucion[numConjActual], pesos)
				pesoConjComparar = peso(particionSolucion[numConjComparar], pesos)
				hago swap de los nodos
				pesoConjActualMod = peso(particionSolucion[numConjActual], pesos)
				pesoConjCompararMod = peso(particionSolucion[numConjComparar], pesos)
				si (pesoConjActual + pesoConjComparar <= pesoConjActualMod + pesoConjCompararMod)
					deshago el swap y coloco los nodos en sus conjuntos originales
					posConjComparar++
				si no
					modifico solucion con los valores acordes
					posConjComparar++
					posConjActual++
				fin si
			fin mientras
		fin mientras
	fin si
fin funcion
\end{lstlisting}

\textbf{Heurística de Búsqueda Local 2}

La segunda vecindad está sujeta a la siguiente relación, $N(S) = $ quitar un nodo de un conjunto de la particion y colocarlo en otro conjunto de la misma particion. En este caso, no estoy tomando 2 nodos particulares, sino estoy observando el comportamiento de 1 sólo nodo, puesto en otros conjuntos.

Dicho esto, un vecino que mejore el peso total de la solución es aquel tal que quito un nodo de un conjunto, denominado $conjActual$, y lo coloco en otro conjunto, distinguido como $conjAgregar$, tal que el resultado de ese cambio, resultando en $conjActual'$ y $conjAgregar'$, se comporte de acuerdo a la siguiente desigualdad: $peso(conjActual) + peso(conjComparar) > peso(conjActual') + peso(conjComparar')$.

Notar que:

\begin{itemize}
\item Cada vez que encuentro un vecino favorable, que mejore el peso total, debo volver a revisar desde el ''principio'' toda la partición, ya que mi vecindad pide existencia en toda la partición, no desde un conjunto en adelante como la anterior.
\end{itemize}

\textbf{Pseudo-código}

\begin{lstlisting}
void busqueda_local_2(Solucion solucion, int k, Pesos pesos)
	Particion particionSolucion(k)
	organizar los nodos en particionSolucion acorde a solucion

	si k > 1
		numConjComparar = 1
		numConjActual = 0
		posConjActual = 0

		mientras (numConjActual < k + 1)
			si (termine de recorrer el conjActual)
				buscar en (numConActual+1...k) el proximo conjunto no vacio y colocar su numero en numConjActual
				posDentroConjActual = 0;

				si (numConjActual > k)
					terminar ejecucion
				si no
					numConjComparar = ((numConjActual == 0) ? 1 : 0)
				fin si
			fin si

			mientras (numConjAgregar < k)
				pesoConjActual = peso(particionSolucion[numConjActual], pesos)
				pesoConjAgregar = peso(particionSolucion[numConjAgregar], pesos)
				pesoConjActualMod = pesoConjActual - pesoDeNodoEnConjunto(particionSolucion[numConjActual], nodo actual, pesos)
				pesoConjCompararMod = pesoConjComparar + pesoDeNodoEnConjunto(particionSolucion[numConjAgregar], nodo actual, pesos)

				si(pesoConjActual + pesoConjAgregar <= pesoConjActualMod + pesoConjAgregarMod)
					numConjAgregar++
					si numConjAgregar == numConjActual
						numConjAgregar++
					fin si
				si no
					quitar nodo actual de conjunto actual
					colocar nodo actual en conjunto agregar
					modifico solucion con los valores acordes

					posConjActual = 0
					buscar en (1...k) el proximo conjunto no vacio y colocar su numero en numConjActual

					numConjComparar = ((numConjActual == 0) ? 1 : 0)
				fin si
			fin mientras

			posConjActual++
		fin mientras
	fin si
fin funcion
\end{lstlisting}
\newpage

\subsection{Complejidad}
\begin{itemize}
\item Vector es la estructura de la librería STL de C++, con las correspondientes complejidades para sus operaciones. Los valores float e int, son los provistos, nativamente, por el lenguaje. Definimos: Peso como float, Nodo como int, Pesos como vector(vector(Peso)), conjNodos como vector(Nodo), Partición como vector(conjNodos) y Solución como vector(int).
\item Como referencia, usamos el código fuente que se encuentra al final del documento.
\item Consideramos la línea 1 de cada función como la declaración de la misma (return type, parametros y apertura de llaves). Por ende, cuando se mencione ''el código en la línea x'', se está haciendo referencia al código que se encuentra en la línea x desde la línea 1.
\item Operaciones triviales (pedir el tamaño de un vector, comparaciones entre ints, etc) no van a ser mencionadas en el cálculo de complejidad, excepto en casos donde no sea evidente que tienen la complejidad que declaramos.
\end{itemize}

\textbf{Complejidad Algoritmo de Búsqueda Local 1}

Es posible dar una cota de complejidad del algoritmo ya que la vecindad está definida no sólo en base a la solución, sino también por el conjunto y el nodo que estoy analizando actualmente. En palabras más formales, el tamaño de la vecindad se va reduciendo a medida que el algoritmo encuentra un vecino ''mejor''.

\begin{itemize}
\item Línea 4: Declaración particiónSolución con tamaño k $\in \Theta(k)$
\item Líneas 5 - 7: for de $0$ a $(k-1)$ que hace operación reserve con parámetro $n$, la operación reserve $\in O(n)$, entonces el for $\in O(k*n)$
\item Líneas 9 - 10: for de $0$ a $(n-1)$ que hace push\_back al conjunto dentro de la partición correspondiente, el operator[] $\in O(1)$ mientras que, por haber hecho reserve($n$) previamente, push\_back $\in O(1)$. Complejidad total = $O(n)$.
\item Línea 14 - 23: Declaraciones de ints y floats, y a su vez uso de operación .size(). En conjunto, las operaciones $\in O(1)$.
\item Línea 26 - 96: While que ejecuta su cuerpo $(k-1)$ veces $\in \Theta(k*CuerpoWhile1)$.
\item CuerpoWhile1:
\begin{itemize}
\item Líneas 30 - 32: While que itera $k$ veces, en peor caso, realizando operaciones con complejidad $\in O(1)$ $\Rightarrow \in O(k)$.
\item Líneas 37 - 39: While que itera $k$ veces, en peor caso, realizando operaciones con complejidad $\in O(1)$ $\Rightarrow \in O(k)$.
\item Líneas 53 - 55: While que itera $k$ veces, en peor caso, realizando operaciones con complejidad $\in O(1)$ $\Rightarrow \in O(k)$.
\item Líneas 63 - 64: While que itera $k$ veces, en peor caso, realizando operaciones con complejidad $\in O(1)$ $\Rightarrow \in O(k)$.
\item Líneas 75 - 93: While que ejecuta su cuerpo tantas veces como cantidad de elementos haya en conjComparar y conjActual, cuyos tamaños están acotados por $n \Rightarrow \in O(2n*CuerpoWhile2) = O(n*CuerpoWhile2)$.
\item CuerpoWhile2:
\begin{itemize}
\item Líneas 76 - 81: 4 Operaciones pesoDeConjunto, que $\in O(n^2)$ (ver Complejidad Algoritmos Usados), más complejidad del swap $O(1)$ (ver Complejidad Algoritmos Usados) $\Rightarrow O(4*n^2) + O(1) = O(n^2)$.
\item Líneas 83 - 93: Cualquiera de los 2 caminos del if tiene complejidad $O(1)$.
\item Complejidad Total CuerpoWhile2 $\in O(n^2)$.
\end{itemize}
\item Complejidad Total CuerpoWhile1 $\in O(4k + n*(n^2)) = O(k + n^3)$.
\end{itemize}
\end{itemize}

La función que caracteriza la complejidad del algoritmo $\in O(k*n + n + k*(k + n^3)) = O(k*n + k^2 + n^3)$ que, por propiedades de las funciones de orden:

- Si $(k <= n) \Rightarrow O(k*n + k^2 + n^3) = O(n^2 + k^2 + n^3) = O(n^2 + n^2 + n^3) = O(n^3)$ que es polinomial en $n$.

- Si $(k > n) \Rightarrow O(k*n + k^2 + n^3) = O(k^2 + k^2 + k^2*n) = O(n * k^2)$ que es pseudo polinomial ya que no hay una clara relación entre $n$ y $k$.

\textbf{Complejidad Iteración Algoritmo de Búsqueda Local 2}

Para este algoritmo, el análisis de complejidad va a estar sujeto a una iteración del ciclo principal (Líneas 23 - 74) del mismo ya que no hay una cota clara en cuanto a la cantidad de iteraciones del mismo. Una vez que encuentro un conjunto tal que poniendo el nodo actual mejoro el peso total, debo volver a recorrer todos los conjuntos. En otras palabras, la vecindad, dada por el 2do algoritmo de búsqueda local, es más amplia que el primero y no sufre disminuciones en su tamaño.

\begin{itemize}
\item Líneas 26 - 28: While que itera $k$ veces, en peor caso, realizando operaciones con complejidad $\in O(1)$ $\Rightarrow \in O(k)$.
\item Líneas 44 - 45: 2 Operaciones pesoDeConjunto, que $\in O(n^2)$ (ver Complejidad Algoritmos Usados) $\Rightarrow O(2*n^2) = O(n^2)$.
\item Líneas 46 - 47: 2 Operaciones pesoDeNodoEnConjunto, que $\in O(n)$ (ver Complejidad Algoritmos Usados) $\Rightarrow O(2*n) = O(n)$.
\item Línea 56: Operación push\_back() y operator[], ambos con complejidad O(1), complejidad de la línea $\in O(1)$.
\item Línea 57: Operación erase() tiene complejidad lineal en la cantidad de elementos que hay en las posiciones siguientes al último elemento borrado, el peor caso es borrar el primero, $\Rightarrow \in O(n)$ (si para mover los elementos realiza copias, no hay problema ya que copiar ints es $O(1)$).
\item Línea 63 - 65: While que itera $k$ veces, en peor caso, realizando operaciones con complejidad $\in O(1)$ $\Rightarrow \in O(k)$.
\end{itemize}

Habiendo analizado todo lo relevante, en cuánto a complejidad temporal, del código, observamos que la función que caracteriza la complejidad de una iteración del ciclo principal $\in O(2k + (n^2) + n) = O(k + n^2)$ que, por propiedades de las funciones de orden:

- Si $(k <= n) \Rightarrow O(k + n^2) = O(n^2)$ que es polinomial en $n$.

- Si $(k > n) \Rightarrow O(k + n^2) = O(k + k^2) = O(k^2)$ que es pseudo polinomial ya que no hay una clara relación entre $n$ y $k$.

\subsection{Experimentación}

Por problemas nuestros, estuvimos cortos de tiempo para hacer esta experimentaciòn. Espero sepan disculpar.

% Fin Ejercicio 1

\newpage
% Comienzo Ejercicio 2
\section{Metahurística GRASP}
\subsection{Desarrollo}
El problema de K-PMP es un problema muy dificil de resolver, debido a que es NP-Completo le toma mucho tiempo a un algoritmo exacto resolverlo, se puede ver facilmente que para generar todas las particiones posibles, esto no es polinomial (ver ejercicio 2).

Utilizando una \textit{heuristica constructiva golosa}, se lo pudo resolver en tiempo polinomial aunque obteniendo en su mayor parte soluciones suboptimas  (ver ejercicio 3).

Utilizando una \textit{heuristica de busqueda local}, se pudieron proponer diferentes vecindades para soluciones, este algoritmo de busqueda local es capaz de devolvernos la mejor distribucion de una vecindad dada su solucion, esto es polinomial, pero aun estamos lejos de obtener el valor exacto (ver ejercicio 4).

Lo que se nota utilizando cualquiera de las dos heuristicas anteriormente planteadas entonces es, estamos realizando diferentes estrategias y criterios para tratar de resolver el mismo problema, \textit{mejorando en complejidad temporal pero siempre  sacrificando presicion de la solucion} al problema propiamente dicho.

El objetivo ahora es encontrar una manera de combinar ambas heuristicas, y eventualmente poder llegar a la solucion optima, o al menos poder acercarlo lo suficientemente, sin tener que sacrificar tanto tiempo para su resolucion: La \textbf{Metahuristica GRASP (Greedy Randomized Adaptive Search Procedures)} cumple efectivamente.

Feo y Resende explican como la efectividad de la busqueda local depende de varios factores: la estructura de la vecindad, la funcion a ser minimizada y la solucion con la que se empieza. Una solucion se dice que esta en la "cuenca de atraccion" del optimo global si es que la busqueda local arrancando con dicha solucion, lleva al optimo global.

Una vez que tenemos los criterios de la busqueda local (se eligio el primer criterio, ejercicio 4), lo unico que nos hace falta es diferentes soluciones con las cuales arrancar, e ir probandolas hasta poder dar con alguna que este en la cuenca de atraccion, o al menos se acerque al valor de la solucion optima lo suficientemente sin sacrificar gran cantidad de tiempo.

Utilizar soluciones aleatorias son de calidad pobre en general, la heuristica golosa produce mejores soluciones que las aleatorias, aunque suboptimas: se elije al elemento mejor posicionado, y se lo agrega a la construccion. Esta heuristica siempre vendria a generar la misma solucion, haciendo que si la solucion final no cae en la cuenca de atraccion, nunca se llegaria al optimo global utilizando la busqueda local.

Es por eso entonces que se usa una heuristica golosa \textbf{aleatorizada} (RCL, Restricted Candidate List), lo que le agrega variacion a esta construccion de solucion golosa, en vez de siempre elegir el mejor posicionado, se lo agrega a una lista de candidatos para la construccion en esa iteracion, y de todos ellos se elige uno al azar.

Para construir este RCL, se eligio el esquema basado en el \textbf{valor}: todos los elementos candidatos con funcion golosa (en este caso maximo peso) dentro de un $\alpha$\% del valor goloso, son colocados en el RCL.

\begin{lstlisting}
Candidato function pickOne(vector<Candidato> candidatosOrdenados, var alpha)
	var limit = ceil(alpha * candidatosOrdenados.size())
	var i = rand(0, 100) % (int)limit
	Retornar y eliminar el i-esimo candidato de candidatosOrdenados
Fin
\end{lstlisting}

Ejecutamos entonces este procedimiento hasta cumplir un criterio de parada: Que la solucion f* no mejore luego de k iteraciones, que no mejore en un un $\sigma$\% luego de k iteraciones, que se repitan varias veces las mismas soluciones golosas, o que simplemente se lo deje ejecutando hasta que se decida cortarlo.

\begin{lstlisting}
f* = MAX_VALUE
Repetir hasta que el criterio de parada este cumpla:

	Generar solucion golosa x
	Busqueda local de un optimo local x', arrancando desde x
	Si f(x') < f*
		f* = f(x')
		x* = x'
	Fin Si

Fin
\end{lstlisting}

% Fin Ejercicio 2

%\newpage
% Comienzo Ejercicio 3
%\input{ejercicio6}
% Fin Ejercicio 3

\end{document}